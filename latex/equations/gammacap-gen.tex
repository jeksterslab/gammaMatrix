\begin{equation}
    \label{eq:gammacap-gen}
    \hat{\boldsymbol{\Gamma}}
    =
    n^{- 1}
    \sum_{i = 1}^{n}
    \left\{
      \mathrm{vech}
      \left[
      \left(
      \mathbf{X}_i - \hat{\boldsymbol{\mu}}
      \right)
      \left(
      \mathbf{X}_i - \hat{\boldsymbol{\mu}}
      \right)^{\prime}
      -
      \hat{\boldsymbol{\Sigma}}
      \right]
      \mathrm{vech}^{\prime}
      \left[
      \left(
      \mathbf{X}_i - \hat{\boldsymbol{\mu}}
      \right)
      \left(
      \mathbf{X}_i - \hat{\boldsymbol{\mu}}
      \right)^{\prime}
      -
      \hat{\boldsymbol{\Sigma}}
      \right]
      \right\}
\end{equation}

\noindent where
$\mathbf{X}$ is the $n \times p$ matrix of sample data,
$n$ is the sample size,
$p$ is the number of variables in $\mathbf{X}$,
$\hat{\boldsymbol{\mu}}$ is the $p \times 1$ vector of means of $\mathbf{X}$,
$\hat{\boldsymbol{\Sigma}}$ is the covariance matrix of $\mathbf{X}$,
$\mathrm{vech}$ is the half-vectorization operator,
that is, vector of unique elements of an input symmetric matrix
stacked by column,
$\left( \cdot \right)^{\prime}$
is the transpose operator\footnote{
  See equation on page 496 of \Textcite{Yuan-2006}
  and equation 12 on page 674 of \Textcite{Yuan-2011}.
}.
